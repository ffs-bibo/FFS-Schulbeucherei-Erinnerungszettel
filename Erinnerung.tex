% !TeX encoding = UTF-8
% !TeX spellcheck = de_DE
% !TEX TS-program = lualatex
\ifdefined\directlua\else
    \errmessage{LuaTeX is required to typeset this document}
    \csname @@end\expandafter\endcsname
\fi
\documentclass[
	version=last,
	fontsize=12pt,
	DIV=20,
	parskip=half,
	twoside=semi,
	sfdefaults=true,
	usegeometry
]{scrartcl}
\usepackage[a5paper, landscape, margin=0.8cm]{geometry}
\usepackage{polyglossia}
\setmainlanguage[spelling=new,variant=german]{german}
\usepackage{libertine}
\usepackage{microtype}
\usepackage{fontspec}
\usepackage{tikz}
\usepackage{xcolor}
\usepackage{hyperref}
\usepackage{titlesec}
\setromanfont{Linux Libertine O}
\setsansfont{Linux Biolinum O}
\setmainfont{Linux Biolinum O}
\usetikzlibrary{positioning,fit,calc,spath3}

\definecolor{linkclr}{rgb}{0, 0, 0}
\hypersetup{
	pdfauthor={Oliver Schneider},
	pdftitle={Schulbücherei FFS, Erinnerungszettel},
	pdfsubject={Erinnerungszettel für die Schulbücherei der Friedrich-Fröbel-Schule in Frankfurt/Niederrad},
	pdfkeywords={Bibliothek,Bücherei,Mahnung,Erinnerung}
	unicode,
	bookmarksnumbered=false,
	breaklinks=false,
	colorlinks=true,
	citecolor=black,
	filecolor=black,
	linkcolor=black,
	menucolor=black,
	urlcolor=linkclr,
}

\titleformat{\section}[block]{\normalfont\huge\scshape\bfseries\centering}{\thesection}{1em}{}
\titleformat{\subsection}[block]{\normalfont\large\scshape\bfseries\centering}{\thesubsection}{1em}{}
%\usepackage{lua-visual-debug}
%\usepackage{showframe}

\tikzset{
	debugstyle/.style={
	},
	field/.style={
	inner xsep=0pt,
	font=\large,
	debugstyle,
	},
	fliesstext/.style={
	inner xsep=0pt,
	debugstyle,
	},
	schreiblinie/.style={
		draw,
		dotted,
		dash pattern=on 1pt off 7pt,
		thick,
		gray,
		line cap=round,
	},
	buchdetails/.style={
		draw,
		rounded corners,
		gray,
		thick,
		dashed,
		dash pattern=on 5pt off 4pt,
		text height=6.5cm,
		anchor=north west,
		align=right,
		line cap=round,
	},
}

\begin{document}
\section*{\makebox[1em]{\normalsize\textcolor{gray}{\dotfill}}~Erinnerung}
\vspace*{-1ex}
\subsection*{Schulbücherei der Friedrich-Fröbel-Schule, Niederrad}

\begin{tikzpicture}[text width=\textwidth,anchor=text]
	\begin{scope}[text width=0.7\textwidth,node distance=1ex]
		\node[field] (Datum) {Datum:~\textcolor{gray}{\dotfill}};
		\node[field,below=of Datum] (Adressat) {An:~\textcolor{gray}{\dotfill}};
		\node[field,below=of Adressat] (Klassenlehrer) {\makebox[5.5em]{Klasse:~\textcolor{gray}{\dotfill}}~Lehrer/in:~\textcolor{gray}{\dotfill}};
		\node[field,below=of Klassenlehrer] (Anrede) {Liebe/r~\textcolor{gray}{\dotfill}};
		\node[fit=(Datum)(Adressat)(Klassenlehrer)(Anrede)](Felder) {};
	\end{scope}

	% Bücherwurm
	\begin{scope}[overlay]
		\node[right=0px of Felder.east,yshift=0.5cm] {\includegraphics[width=0.25\textwidth]{bookworm.pdf}};
	\end{scope}

	\begin{scope}[text width=0.39\textwidth,node distance=1ex,anchor=north west]
		\node[fliesstext,below right=of Felder.south west,align=justify]
			(Buchlistenverweis)
			{Du hast folgendes Buch bei uns ausgeliehen und leider nicht rechtzeitig zurückgegeben:};

		\node[fliesstext,below=2ex of Buchlistenverweis,align=justify]
			(Bitte)
			{Bitte bringe das Buch so schnell wie möglich zurück in die Schulbücherei.};

		\node[fliesstext,below=2ex of Bitte,align=justify]
			(Verlaengerung)
			{Falls du es noch nicht fertig gelesen hast, verlängern wir gerne deine Ausleihzeit.};

		\node[fliesstext,below=2ex of Verlaengerung,align=right]
			(Gruss)
			{Dein freundliches Büchereiteam};
		\node[fit=(Buchlistenverweis)(Bitte)(Verlaengerung)(Gruss)](Erinnerungstext) {};
	\end{scope}

	\begin{scope}[text width=0.56\textwidth,node distance=1ex,anchor=north west]
		\node[right=0mm of Erinnerungstext.north east,buchdetails] (Buchdetails) {
			{\scriptsize Buchtitel und Karteinummer, ggf. mehrere auf einmal}
		};
		\coordinate (linestart) at ($(Buchdetails.north west) + (1em, 0)$);
		\coordinate (lineend) at ($(Buchdetails.north east) + (-1em, 0)$);
		% Lines
		\path[schreiblinie] ($(linestart) - (0,1.25cm)$) -- ($(lineend) - (0,1.25cm)$);
		\path[schreiblinie] ($(linestart) - (0,2.5cm)$) -- ($(lineend) - (0,2.5cm)$);
		\path[schreiblinie] ($(linestart) - (0,3.75cm)$) -- ($(lineend) - (0,3.75cm)$);
		\path[schreiblinie] ($(linestart) - (0,5cm)$) -- ($(lineend) - (0,5cm)$);
		\path[schreiblinie] ($(linestart) - (0,6.25cm)$) -- ($(lineend) - (0,6.25cm)$);
	\end{scope}
\end{tikzpicture}
\end{document}
